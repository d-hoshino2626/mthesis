\documentclass[../../main.tex]{subfiles}

\graphicspath{{../../fig/}}
\setcounter{section}{0}

\begin{document}
\begin{center}
\Large\textbf{概要}
\end{center}
\indent 宇宙マイクロ波背景放射(CMB)は宇宙初期から現在に至るまでの宇宙の状態を反映し、
標準宇宙モデルの構築に重要な役割を果たしてきた。
現在はその偏光パターンの測定が注目されている。
その理由はBモードと呼ばれる偏光パターンにはインフレーションによる原始重力波の痕跡が残ると期待されているからである。
Simons Observatory実験では、海抜$\SI{5,200}{m}$のチリ・アタカマ砂漠に複数のCMB望遠鏡を設置しCMB偏光の精密測定を行うことで、
原始重力波の痕跡とされるテンソルスカラー比$r$を$\sigma(r)=0.003$という精度で測定することを目指す。
この精度を実現するためには、検出器の応答する偏光軸の向き(偏光角)を$<0.1\tcdegree$という精度で較正しなければならない。
この要求を満たすため、スパースワイヤーグリッドを用いた偏光角較正装置の開発・配備が行われてきた。
スパースワイヤーグリッドは疎に張られた金属ワイヤーが周囲の熱放射を反射し、
ワイヤーに沿う方向に直線偏光した光を生成する装置である。
スパースワイヤーグリッドが生成する直線偏光を較正源として使う場合にはワイヤーの方向を把握する必要があり、
その把握精度が偏光角の較正精度を左右する。

ワイヤーの向きを天球面上の向きと対応づけるには、スパースワイヤーグリッドが重力となす角度が必要である。
この角度を測定するために2軸の重力参照計を導入した。
本研究では、この重力参照計の精度の評価を行った。
長期的に安定した精度が維持されるか確認するために、1ヶ月の間に出力値がどう変動するか評価した。
また、観測サイトでの$\SI{-20}{\degreeCelsius}$から$\SI{20}{\degreeCelsius}$の激しい温度変化に耐え、
一定の精度を維持するか確認するために、恒温槽を用いて出力の温度変動を評価した。
以上の2つの評価により、2軸のうち1軸は出力角度の精度が$<0.04\tcdegree$と要求性能を満たす結果であった。
もう1軸については初期不良により満たさなかったが、修理することで同じく$<0.04\tcdegree$の精度が期待できる。


偏光角較正に使用する光はワイヤーに沿う方向に直線偏光するため、ワイヤーのたわみも系統誤差の要因となる。
先行研究ではたわみ量の評価を手動にて行っていたため、多大な労力がかかるとともに人依存のバイアスが含まれる可能性があった。
本研究では、ワイヤーのたわみ量を自動で評価する装置を開発した。
これにより、人の手を介さず、以前よりも高精度な$\SI{50}{\micro m}$の精度でのたわみ量の評価が可能となった。
また、実際に使用されるスパースワイヤーグリッドに対してたわみ量の評価を行い、たわみ由来の系統誤差を$0.03\tcdegree$程度に抑えられることを確認した。

以上の系統誤差の評価により、装置由来の偏光角較正の系統誤差が要求精度$<0.1\tcdegree$を満たし得ることを確認した。
\end{document}
