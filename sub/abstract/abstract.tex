\documentclass[../../main.tex]{subfiles}

\graphicspath{{../../fig/}}
\setcounter{section}{0}

\begin{document}
\begin{center}
\Large\textbf{概要}
\end{center}
\indent 

宇宙マイクロ波背景放射(CMB)には宇宙初期から現在に至るまで様々な宇宙の情報が含まれ、
その観測はこれまでに標準宇宙モデルの構築に重要な役割を果たしてきた。
現在はその偏光パターンの測定が注目されている。
その理由はBモードと呼ばれる偏光パターンにはインフレーションによる原始重力波の痕跡が残ると期待されているからである。
Simons Observatory実験では、海抜$\SI{5200}{m}$のチリ・アタカマ砂漠に設置した複数のCMB望遠鏡を用いてCMB偏光の精密測定を行うことで、
原始重力波の検出を目指す。具体的には、その強度指標テンソルスカラー比$r$に対しての誤差$\sigma(r)=0.003$という精度で測定することを目指している。
この精度を実現するためには、検出器の応答する偏光軸の向き(偏光角)を$<0.1\tcdegree$という精度で較正することが要求される。
この要求を満たすため、スパースワイヤーグリッドを用いた偏光角較正装置の開発・配備が行われてきた。
スパースワイヤーグリッドは疎に張られた金属ワイヤーが周囲の熱放射を反射し、
ワイヤーに沿う方向に直線偏光した光を生成する装置である。
スパースワイヤーグリッドが生成する直線偏光を較正源として使う場合にはワイヤーが張られている方向を把握する必要があり、
その把握精度が偏光角の較正精度を左右する。

ワイヤーの向きと天球座標上の向きを対応づけるには、スパースワイヤーグリッドが重力となす角度が必要である。
この角度を測定するために2軸の重力参照計を導入した。本研究では、この重力参照計の精度の評価を行った。
長期的に安定した精度が維持されるか確認するために、1ヶ月間にわたって出力値の安定性を評価した。
また、観測サイトにおける過酷な気温変動($\SI{-20}{\degreeCelsius}$から$\SI{20}{\degreeCelsius}$)のもとで精度を維持することを確認するために、恒温槽を用いて出力の温度変動を評価した。
以上2つの評価により、2軸のうち1軸は出力角度の精度が$<0.04\tcdegree$と要求性能を満たす結果であった。
もう1軸については初期不良により満たさなかったが、修理することで同じく$<0.04\tcdegree$の精度が期待できる。

偏光角の較正において、ワイヤーのたわみも系統誤差の要因となる。
先行研究ではたわみ量の評価を手動にて行っていたため、人依存のバイアスが含まれる可能性とともに多大な労力がかかっていた。
本研究では、ワイヤーのたわみ量を自動で評価する装置を開発し、人の手を介さず以前よりも高精度($\SI{15}{\micro m}$)でのたわみ量の評価が可能となった。
また、実際に使用されるスパースワイヤーグリッドに対してたわみ量の評価を行い、ワイヤーのたわみに由来する偏光角構成の系統誤差を$0.02\tcdegree$程度に抑えられることを確認した。

以上の系統誤差の評価研究により、偏光角較正の系統誤差に対して要求精度($<0.1\tcdegree$)を満たせることを確認した。
\end{document}
