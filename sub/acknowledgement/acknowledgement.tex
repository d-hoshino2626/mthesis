\documentclass[../../main.tex]{subfiles}

\graphicspath{{../../fig/}}
\setcounter{section}{0}

\begin{document}
\chapter*{謝辞}
本修士論文を執筆するにあたり、多くの方にご支援いただきました。この場をお借りして、心より感謝申し上げます。


指導教員である田島治教授には、スパースワイヤーグリッドという魅力的な研究テーマにお誘いいただき、
研究内容や発表に関して多くのご指導をいただきました。
添削されたスライドや論文が、一読するだけで研究の方向性や内容が理解できるようになっていたことに、常々驚かされておりました。
また、観測サイトであるチリへの渡航をサポートしていただき、普通では得られない貴重な経験をさせていただきました。
このような経験は、私の今後の人生において大きな財産となると確信しております。

副指導教官である安達俊介特別助教には、研究内容の隅々に至るまで、非常に多くのアドバイスをいただきました。
機器の読み出しや解析手法、CADの取り扱い方など、研究に必要な技術を習得するためのサポートをしていただいただけでなく、
研究の方向性や進め方について、気さくに相談に乗っていただきました。
時折する雑談も、私にとっては研究を続けていく上での励みとなっていました。
安達さんのご指導がなければ、修士論文を執筆することはできなかったと思います。本当にありがとうございました。

鈴木惇也助教にもお世話になりました。
普段から私が細かく研究内容を共有していないにも関わらず、
グループミーティングやコーヒーブレイクなどの場でちょろっとこぼした内容に対して
適切なアドバイスをいただくことが何度もあり、その洞察力にはいつも驚かされていました。

日本Simons Observatoryグループの皆様にも感謝いたします。
東京大学の日下暁人と木内健司助教には、ミーティング中に適切な助言をいただき、研究内容を深めることができました。

京都CMBグループの皆様にも感謝いたします。
特に、中田嘉信氏はスパースワイヤーグリッドの開発者であったこともあり、私にさまざまなことを教えていただきました。
説明の際に設計思想を伝えていただいたことで、素早くスパースワイヤーグリッドへの理解を深めることができました。
また、自分の研究がどんな物理につながっていくのかを再認識する一言をかけていただいたことで、研究に対するモチベーションを保つことができていたと思います。
このような人のことを考えて行動する姿勢は、私にとっても見習うべき点であると強く感じています。
また、竹内広樹さんには研究に対する助言のみならず、一緒に電波天文学ゼミを行ったり、
コーディングに対する姿勢を教えていただいたりと、多くのことを学ばせていただきました。
奥本成美さんには、M1としてCMBグループに入ってきたばかりであったにも関わらず、
スパースワイヤーグリッドのたわみの測定のための解析手法を実装していただいたりと、多くのことをサポートしていただきました。

京都高エネグループの皆様にも感謝いたします。
特に同期の大谷尚輝くん、笠井優太郎くん、片岡敬涼くん、中川徹郎くん、埴村圭吾くんにはお世話になりました。
研究内容はもちろん、日常のどうでもいいことを気兼ねなく話すことができて、とても楽しい研究生活を送ることができました。
君たちの研究に向かう姿であったり、研究室での雑談での発言であったりが、私の心の支えになっていましたことは間違いありません。
その他の皆様についても、この場ではとても書ききれませんが、皆様と共に過ごす研究生活は私にとって非常に有意義なものでした。

最後に、私がここまで研究生活を送ることを応援し、支えてくれた家族に感謝いたします。

\end{document}