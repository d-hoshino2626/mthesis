\documentclass[../../main.tex]{subfiles}

\graphicspath{{../../fig/}}
\setcounter{section}{0}

\begin{document}

\chapter{宇宙マイクロ波背景放射(CMB)}
宇宙マイクロ波背景放射(Cosmic Microwave Background: CMB)とは、宇宙の創生から38万年後に物質から脱結合した光子のことであり、我々が観測できる最古の光である。
その発見はペンジアスとウィルソンによって1965年に行われ\cite{1965ApJ...142..419P}、
その後Cosmic Background Explorer(COBE)衛星により強度の周波数依存性(スペクトル)が測定された\cite{1996ApJ...473..576F}。
測定されたスペクトルは温度が$\SI{2.725}{K}$の黒体輻射のスペクトルと一致し、CMBがほとんど一様等方な強度を持つことも確認された。
これらの事実によりCMBはビッグバン宇宙モデルを支持する強力な証拠となった。
現在ではCMBの偏光揺らぎに注目が集まっており、その偏光情報からインフレーション宇宙論の検証

\section{$\Lambda\mathrm{CDM}$モデル}
ほげほげ
\section{インフレーション宇宙論}

\section{CMB偏光モード}
\section{本論文の構成}

\end{document}