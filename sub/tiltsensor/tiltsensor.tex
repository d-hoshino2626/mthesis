\documentclass[../../main.tex]{subfiles}

\graphicspath{{../../fig/}}
\setcounter{section}{0}

\begin{document}
\chapter{重力参照計の評価}
\ref{}にて述べたように、本較正装置では重力参照計を用いて絶対角度を測定する。
その要求性能としては
\begin{itemize}
    \item 他の系統誤差の評価値を加味して、測定精度が $<0.06\tcdegree$ であること
    \item 電源の入れ直しによるオフセット変動が小さいこと
    \item 観測サイトの環境 $\SI{-15}{\degreeCelsius} \sim \SI{20}{\degreeCelsius}$ にて、温度変動による出力の変化が小さいこと
\end{itemize}
が挙げられる。
これまでに使用が想定されていた Digi-Pas 社製の DWL5000-XY は要求性能を満たさなかったため、
新たに候補となった Sherborne Sensors 社製の DSIC-2051-60 を評価した。
本章では、はじめに新しい重力参照計について共有した後、評価手法と結果について述べる。
\subsection{重力参照計の概要}
図\ref{}に重力参照計の外観を示す。
\section{電源の入れ直しによるオフセット変動の評価}
\subsection{評価系}
\subsection{測定結果}
\section{温度による出力の変化の評価}
\subsection{評価系の概要}
\subsection{測定結果}
\subsection{測定結果の考察}
\subsection{まとめ}
\end{document}