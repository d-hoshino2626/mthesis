\documentclass[../../main.tex]{subfiles}

\graphicspath{{../../fig/}}
\setcounter{section}{0}

\begin{document}
\chapter{ストークスパラメータ}
\label{chap:stokes}
\section{ストークスパラメータの定義}
\label{sec:stokes}
$xyz$ 直交座標系において、$z$ 軸正の方向に進行する光を考える。
この光の電場ベクトルを $\bm{E}(t)$ とすると、これは一般に
\begin{align}
    \bm{E}(t) &= E_{x} \bm{\hat{e}}_{x} + E_{y} \bm{\hat{e}}_{y} \\
        &= a_{x}\cos(\omega t - kz + \delta_x) \bm{\hat{e}}_{x} + a_{y}\cos(\omega t - kz + \delta_y) \bm{\hat{e}}_{y}
\end{align}
と表される。ここで、$E_{x}, E_{y}$ は $x$ 軸、$y$ 軸方向成分であり、
$a_{x}, a_{y}$ は振幅、$\omega$ は角振動数、$k$ は波数、$\delta_x, \delta_y$ は位相オフセットである。
また、$\bm{\hat{e}}_x,\,\bm{\hat{e}}_y$ はそれぞれ $x$ 軸、$y$ 軸正の方向の単位ベクトルである。
$\delta = \delta_y - \delta_x$ とすると、ストークスパラメータ $I,\,Q,\,U,\,V$ は
\begin{align}
    I &= a_{x}^{2} + a_{y}^{2} \\
    Q &= a_{x}^{2} - a_{y}^{2} \\
    U &= 2a_{x}a_{y}\cos\delta \\
    V &= 2a_{x}a_{y}\sin\delta
\end{align}
と定義される。
複素電場ベクトル $\bm{\mathcal{E}}$ を用いてストークスパラメータを表すことを考える。
$\bm{\mathcal{E}}$ は一般に
\begin{align}
    \bm{\mathcal{E}} &= a_{x}e^{i\qty(\omega t-kz+\delta_x)}\bm{\hat{e}}_{x} + a_{y}e^{i\qty(\omega t-kz+\delta_y)}\bm{\hat{e}}_{y} \\
    &= \mathcal{E}_{x}\bm{\hat{e}}_{x} + \mathcal{E}_{y}\bm{\hat{e}}_{y}
    \label{eq:complex_electric_field}
\end{align}
と表される。ここで、$\mathcal{E}_{x},\,\mathcal{E}_{y}$ は $x$ 軸、$y$ 軸方向成分である。
ストークスパラメータは
\begin{align}
    \label{eq:stokes_I_complex}
    I &= \abs{\mathcal{E}_{x}}^{2} + \abs{\mathcal{E}_{y}}^{2} \\
    \label{eq:stokes_Q_complex}
    Q &= \abs{\mathcal{E}_{x}}^{2} - \abs{\mathcal{E}_{y}}^{2} \\
    \label{eq:stokes_U_complex}
    U &= 2\Re\qty[\mathcal{E}_{x}\mathcal{E}_{y}^{*}] = \mathcal{E}_x\mathcal{E}_{y}^{*}+\mathcal{E}_{x}^{*}\mathcal{E}_{y} \\
    \label{eq:stokes_V_complex}
    V &= 2\Im\qty[\mathcal{E}_{x}\mathcal{E}_{y}^{*}] = \dfrac{1}{i}\qty(\mathcal{E}_x\mathcal{E}_{y}^{*}-\mathcal{E}_{x}^{*}\mathcal{E}_{y})
\end{align}
と表される。
\section{座標回転に対するストークスパラメータの変換}
\label{sec:stokes_transform}
$xyz$ 直交座標系を $z$ 軸周りに $\theta$ だけ回転させた $x'y'z$ 直交座標系を考える。
このとき、$x',\,y',\,z$ 軸の単位ベクトル $\bm{\hat{e}}_{x'},\,\bm{\hat{e}}_{y'},\,\bm{\hat{e}}_{z}$ はそれぞれ
\begin{equation}
    \mqty(\bm{\hat{e}}_{x},\,\bm{\hat{e}}_{y},\bm{\hat{e}}_{z}) = 
        \mqty(\bm{\hat{e}}_{x'},\,\bm{\hat{e}}_{y'},\bm{\hat{e}}_{z})
        \mqty(\cos\theta & -\sin\theta & 0 \\
              \sin\theta & \cos\theta & 0 \\
              0 & 0 & 1
              )
\end{equation}
と表される。複素電場ベクトル\eqref{eq:complex_electric_field}を $x'y'z$ 直交座標系で表すと
\begin{align}
    \bm{\mathcal{E}} &= \mathcal{E}_{x}\qty(\cos\theta \bm{\hat{e}}_{x'} + \sin\theta \bm{\hat{e}}_{y'}) + \mathcal{E}_{y}(-\sin\theta \bm{\hat{e}}_{x'} + \cos\theta\bm{\hat{e}}_{y'}) \\
    &= \qty(\mathcal{E}_{x}\cos\theta - \mathcal{E}_{y}\sin\theta)\bm{\hat{e}}_{x'} + \qty(\mathcal{E}_{x}\sin\theta + \mathcal{E}_{y}\cos\theta)\bm{\hat{e}}_{y'} \\
    &= \mathcal{E}_{x'}\bm{\hat{e}}_{x'} + \mathcal{E}_{y'}\bm{\hat{e}}_{y'}
\end{align}
となる。ここで、$\mathcal{E}_{x'} = \mathcal{E}_{x}\cos\theta - \mathcal{E}_{y}\sin\theta,\,\mathcal{E}_{y'}=\mathcal{E}_{x}\sin\theta + \mathcal{E}_{y}\cos\theta$ である。
このとき、$x'y'z$ 直交座標系でのストークスパラメータは
% \begin{align}
%     I'
%     \begin{split}
%         & =\abs(\mathcal{E}_{x})^{2} + \abs(\mathcal{E}_{y})^{2} \\
%         &= \abs(\cos\theta)
%     \end{split} \\
%     aaa
% \end{align}
\begin{align}
    \begin{split}
        I' &= \abs{\mathcal{E}_{x'}}^{2} + \abs{\mathcal{E}_{y'}}^{2} \\
           &= \abs{\mathcal{E}_{x}\cos\theta - \mathcal{E}_{y}\sin\theta}^{2} + \abs{\mathcal{E}_{x}\sin\theta + \mathcal{E}_{y}\cos\theta}^{2} \\
           &= \abs{\mathcal{E}_{x}}^{2} + \abs{\mathcal{E}_{y}}^{2} \\
           &= I
    \end{split} \\
    \begin{split}
        Q' &= \abs{\mathcal{E}_{x'}}^{2} - \abs{\mathcal{E}_{y'}}^{2} \\
           &= \abs{\mathcal{E}_{x}\cos\theta - \mathcal{E}_{y}\sin\theta}^{2} - \abs{\mathcal{E}_{x}\sin\theta + \mathcal{E}_{y}\cos\theta}^{2} \\
           &= \qty(\abs{\mathcal{E}_{x}}^{2} - \abs{\mathcal{E}_{y}}^{2})\cos2\theta - \qty(\mathcal{E}_{x}\mathcal{E}_{y}^{*} + \mathcal{E}_{x}^{*}\mathcal{E}_{y})\sin2\theta \\
           &= Q\cos2\theta - U\sin2\theta
    \end{split} \\
    \begin{split}
        U' &= \mathcal{E}_{x'}\mathcal{E}_{y'}^{*} - \mathcal{E}_{x'}^{*}\mathcal{E}_{y'} \\
           &= (\mathcal{E}_{x}\cos\theta - \mathcal{E}_{y}\sin\theta)(\mathcal{E}_{x}\sin\theta + \mathcal{E}_{y}\cos\theta)^{*} \\
                &\hspace{2cm} - (\mathcal{E}_{x}\cos\theta - \mathcal{E}_{y}\sin\theta)^{*}(\mathcal{E}_{x}\sin\theta + \mathcal{E}_{y}\cos\theta) \\
           &= \qty(\abs{\mathcal{E}_{x}}^{2} - \abs{\mathcal{E}_{y}}^{2})\sin2\theta + \qty(\mathcal{E}_{x}\mathcal{E}_{y}^{*} + \mathcal{E}_{x}^{*}\mathcal{E}_{y})\cos2\theta \\
           &= Q\sin2\theta + U\cos2\theta
    \end{split} \\
    \begin{split}
        V' &= \dfrac{1}{i}\qty(\mathcal{E}_{x'}\mathcal{E}_{y'}^{*} - \mathcal{E}_{x'}^{*}\mathcal{E}_{y'}) \\
              &= \dfrac{1}{i}\left[(\mathcal{E}_{x}\cos\theta - \mathcal{E}_{y}\sin\theta)(\mathcal{E}_{x}\sin\theta + \mathcal{E}_{y}\cos\theta)^{*} \right. \\
              &\hspace{2cm} \left.- (\mathcal{E}_{x}\cos\theta - \mathcal{E}_{y}\sin\theta)^{*}(\mathcal{E}_{x}\sin\theta + \mathcal{E}_{y}\cos\theta) \right] \\
           &= \dfrac{1}{i}\qty(\mathcal{E}_{x}\mathcal{E}_{y}^{*} - \mathcal{E}_{x}^{*}\mathcal{E}_{y}) \\
           &= V
    \end{split}
\end{align}
となる。すなわち
\begin{equation}
    \mqty(I' \\ Q' \\ U' \\ V') = \mqty(1 & 0 & 0 & 0 \\ 0 & \cos2\theta & -\sin2\theta & 0 \\ 0 & \sin2\theta & \cos2\theta & 0 \\ 0 & 0 & 0 & 1)\mqty(I \\ Q \\ U \\ V)
    \label{eq:stokes_transform}
\end{equation}
と座標の回転に対して2倍の角度で回転することがわかる。
\end{document}