\documentclass[../../main.tex]{subfiles}

\graphicspath{{../../fig/}}
\setcounter{section}{0}

\begin{document}
\chapter{ストークスパラメータ}
\label{chap:stokes}
\section{ストークスパラメータの定義}
\label{sec:stokes}
$xyz$ 直交座標系において、$z$ 軸正の方向に進行する光を考える。
この光の電場ベクトルを $\bm{E}(t)$ とすると、これは一般に
\begin{align}
    \bm{E}(t) &= E_{x} \bm{\hat{e}}_{x} + E_{y} \bm{\hat{e}}_{y} \\
        &= a_{x}\cos(\omega t - kz + \delta_x) \bm{\hat{e}}_{x} + a_{y}\cos(\omega t - kz + \delta_y) \bm{\hat{e}}_{y}
\end{align}
と表される。ここで、$E_{x}, E_{y}$ は振幅、$\omega$ は角振動数、$k$ は波数、$\delta_x, \delta_y$ は位相オフセットであり、
$\bm{\hat{e}}_x,\,\bm{\hat{e}}_y$ はそれぞれ $x$ 軸、$y$ 軸正の方向の単位ベクトルである。
$\delta = \delta_y - \delta_x$ とすると、ストークスパラメータ $I,\,Q,\,U,\,V$ は
\begin{align}
    I &= a_{x}^{2} + a_{y}^{2} \\
    Q &= a_{x}^{2} - a_{y}^{2} \\
    U &= 2a_{x}a_{y}\cos\delta \\
    V &= 2a_{x}a_{y}\sin\delta
\end{align}
と定義される。
複素電場ベクトル $\bm{\mathcal{E}}$ を用いてストークスパラメータを表すことを考える。
$\bm{\mathcal{E}}$ は一般に
\begin{align}
    \bm{\mathcal{E}} &= \mathcal{E}_{x}\bm{\hat{e}}_{x} + \mathcal{E}_{y}\bm{\hat{e}}_{y} \\
    &= a_{x}e^{i\qty(\omega t-kz+\delta_x)}\bm{\hat{e}}_{x} + a_{y}e^{\qty(\omega t-kz+\delta_y)}\bm{\hat{e}}_{y}
    \label{eq:complex_electric_field}
\end{align}
と表される。ここで、$\mathcal{E}_{x}, \mathcal{E}_{y}$ は複素振幅である。
ストークスパラメータは
\begin{align}
    I &= \abs{\mathcal{E}_{x}}^{2} + \abs{\mathcal{E}_{y}}^{2} \\
    Q &= \abs{\mathcal{E}_{x}}^{2} - \abs{\mathcal{E}_{y}}^{2} \\
    U &= 2\Re\qty[\mathcal{E}_{x}\mathcal{E}_{y}^{*}] = \mathcal{E}_x\mathcal{E}_{y}^{*}+\mathcal{E}_{x}^{*}\mathcal{E}_{y} \\
    V &= 2\Im\qty[\mathcal{E}_{x}\mathcal{E}_{y}^{*}] = \mathcal{E}_x\mathcal{E}_{y}^{*}-\mathcal{E}_{x}^{*}\mathcal{E}_{y}
\end{align}
と表される。
\section{座標変換に対するストークスパラメータの変換}
\label{sec:stokes_transform}
$xyz$ 直交座標系を $z$ 軸周りに $\theta$ だけ回転させた $x'y'z$ 直交座標系を考える。
このとき、$x',y',z$ 軸の単位ベクトル $\bm{\hat{e}}_{x'},\bm{\hat{e}}_{y'},\bm{\hat{e}}_{z}$ はそれぞれ
\begin{equation}
    \mqty(\bm{\hat{e}}_{x},\,\bm{\hat{e}}_{y},\bm{\hat{e}}_{z}) = 
        \mqty(\bm{\hat{e}}_{x'},\,\bm{\hat{e}}_{y'},\bm{\hat{e}}_{z})
        \mqty(\cos\theta & -\sin\theta & 0 \\
              \sin\theta & \cos\theta & 0 \\
              0 & 0 & 1
              )
\end{equation}
と表される。複素電場ベクトル\eqref{eq:complex_electric_field}を $x'y'z$ 直交座標系で表すと
\begin{align}
    \bm{\mathcal{E}} &= \mathcal{E}_{x}e^{i\qty(\omega t-kz+\delta_x)}\qty(\cos\theta \bm{\hat{e}}_{x'} + \sin\theta \bm{\hat{e}}_{y'}) \notag \\
        &\hspace{3cm} + \mathcal{E}_{y}e^{\qty(\omega t-kz+\delta_y)}(-\sin\theta \bm{\hat{e}}_{x'} + \cos\theta\bm{\hat{e}}_{y'}) \\
    &= \qty(\mathcal{E}_{x}\cos\theta - \mathcal{E}_{y}\sin\theta)e^{i\qty(\omega t-kz+\delta_x)}\bm{\hat{e}}_{x'} \notag \\
        &\hspace{3cm} + \qty(\mathcal{E}_{x}\sin\theta + \mathcal{E}_{y}\cos\theta)e^{i\qty(\omega t-kz+\delta_y)}\bm{\hat{e}}_{y'}
\end{align}
となる。このとき、$x'y'z$ 直交座標系でのストークスパラメータは
% \begin{align}
%     \begin{split}
%         I' &= \abs(\mathcal{E}_{x})^{2} + \abs(\mathcal{E}_{y})^{2} \\
%         &= \abs(\cos\theta)
%     \end{split}
%     aaa
% \end{align}

\end{document}