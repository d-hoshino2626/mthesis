\documentclass[../../main.tex]{subfiles}

\graphicspath{{../../fig/}}
\setcounter{section}{0}

\begin{document}
\chapter{まとめと今後の展望}
\section{まとめ}
宇宙マイクロ波背景放射(CMB)の温度揺らぎは、我々に$\Lambda \mathrm{CDM}$モデルという宇宙の進化を記述するモデルを与えた。
さらなる宇宙論の発展として、ビッグバンよりも以前に空間の指数関数的膨張が起こったとするインフレーションモデルが存在する。
CMBの偏光パターンを精査し、原始重力波由来のBモード偏光パターンを発見することは、インフレーションの直接的証拠となる。

地上CMB偏光観測実験であるSimons Observatory実験は、このBモードをテンソルスカラー比にして$\sigma(r)=0.003$の精度で観測することを目指している。
この精度を達成するためには、検出器の応答偏光方向(偏光角)を$\delta\theta<0.1\tcdegree$という精度で較正する必要がある。
そのために使用されるのが、スパースワイヤーグリッドを用いた偏光角較正装置である。
スパースワイヤーグリッドは金属ワイヤーが周囲の熱放射を反射することで人工的に直線偏光を作り出す装置である。
この直線偏光を検出器で見ることによって偏光角を較正する。

本研究では、絶対角度較正に用いられる2軸$(X,\,Y\text{軸})$の重力参照計の精度を長期間の出力安定性という点と、
$\SI{-20}{\degreeCelsius}\sim\SI{20}{\degreeCelsius}$の温度変化に対する出力安定性という点を評価した。
その結果、重力参照計の$X$軸の出力角度$\theta_{X}$の精度が$\delta\theta_{X}<0.04$であると結論づけられた。
$Y$軸の出力については温度変化に対して$\delta\theta_{Y}>1\tcdegree$もの変動が見られたが、これは初期不良によるものであると結論づけられた。
そのため、この不良がなければ$\theta_{Y}$は$\theta_{X}$と同程度の性能は持つだろうと仮定し、$\delta\theta_{Y}<0.04\tcdegree$を見込んでいる。
また、これらの重力参照計の精度を偏光角較正の精度に焼き直すと、望遠鏡が較正を行う基本姿勢$\mathrm{elevation}=50\tcdegree,\,\mathrm{boresight}=0\tcdegree$では
$<0.06\tcdegree$と評価された。

また、系統誤差の要因の1つにワイヤーのたわみがある。
先行研究ではすべての評価を人力で行っており、多大な労力がかかっていたとともにその評価結果に人依存のバイアスを含む可能性があった。
この現状を受け、本研究ではワイヤーのたわみを自動で評価する装置を開発した。
たわみの長さが既知のワイヤーを測定することで評価精度として$<\SI{50}{\micro m}$を得た。
これは装置の光学系を考えた際に得る誤差とよく一致する。
さらに、この評価装置を用いて、実際に使用されるスパースワイヤーグリッドのたわみを評価した。
その評価結果はたわみによる偏光角への影響$\theta_{\mathrm{sag}}$にして、$\theta_{\mathrm{sag}}<0.03\tcdegree$であった。

表\ref{tab:summary_systematic_error}に、本研究で評価した系統誤差の値を更新した、スパースワイヤーグリッドの装置由来の系統誤差をまとめる。
偏光角較正の合計の系統誤差$\delta\theta$は$\delta\theta=0.084\tcdegree<0.09\tcdegree$であった。
これはSimons Observatoryにおける要求精度$\delta\theta<0.1\tcdegree$を満たしている。
\begin{table}
    \centering
    \caption{}
    \begin{tabular}{ccc}
        \hline\hline
        系統誤差要因 & 評価値 \\
        \hline
        ワイヤー設置精度 & $<0.02\tcdegree$ \\
        エンコーダ精度 & $<0.03\tcdegree$ \\
        エンコーダ零点 & $<0.04\tcdegree$ \\
        重力参照計\footnote{これは望遠鏡が基本姿勢$\mathrm{elevation}=50\tcdegree,\,\mathrm{boresight}=0\tcdegree$をとっているときの値であることに注意。} & $<0.06\tcdegree$ \\
        ワイヤーのたわみ & $<0.03$ \\
        \hline
        合計 & $<0.09$ \\
        \hline
        \hline
    \end{tabular}
    \label{tab:summary_systematic_error}
\end{table}
\section{今後の展望}
今後の展望として、まず第一に初期不良で正常に動作していなかった重力参照計の$Y$軸の再評価が予定されている。

\end{document}