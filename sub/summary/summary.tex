\documentclass[../../main.tex]{subfiles}

\graphicspath{{../../fig/}}
\setcounter{section}{0}

\begin{document}
\chapter{まとめと今後の展望}
\section{まとめ}
宇宙マイクロ波背景放射(CMB)の温度揺らぎの精密観測によって宇宙論に対する深い理解が進んだ。
さらなる宇宙論の発展として、ビッグバンよりも以前に空間の指数関数的膨張が起こったとするインフレーションモデルの検証が重要である。
そして、CMBの偏光パターンを精査し原始重力波由来のBモード偏光パターンを発見することは、インフレーションの直接的証拠となる。

地上CMB偏光観測実験であるSimons Observatory実験は、このBモードをテンソルスカラー比$r$を$0.003$の精度で観測することを目指している。
この精度を達成するためには、検出器の応答偏光方向(偏光角)を$\delta\theta<0.1\tcdegree$という精度で較正する必要がある。
そのために使用されるのが、スパースワイヤーグリッドを用いた偏光角較正装置である。
スパースワイヤーグリッドは金属ワイヤーが周囲の熱放射を反射することで人工的にワイヤー方向に沿った直線偏光を作り出す装置である。
この直線偏光を検出器で見ることによって偏光角を較正する。

偏光角を高精度に較正するためには、ワイヤーの角度に関する系統誤差を抑える必要がある。
先行研究での知見をもとに、本研究ではワイヤーの天球座標で対応する角度を知るための重力参照計の精度と、
ワイヤーのたわみの評価の改善策の検証を行い、それらの系統誤差の抑制を試みた。

絶対角度較正のためにより高精度な2軸$(X\text{軸},\,Y\text{軸})$の重力参照計を導入し、その精度を長期間の出力安定性という点と、
$\SI{-20}{\degreeCelsius}\sim\SI{20}{\degreeCelsius}$の温度変化に対する出力安定性という点において評価した。
その結果、重力参照計の$X$軸の出力角度$\theta_{X}$の精度が$\delta\theta_{X}<0.04$であると結論づけられた。
$Y$軸の出力については、残念ながら徹底的な評価実験によって初期不良であるとわかった。
現在、メーカーにより重力参照計の修理が行われており、この不良が治れば$\theta_{Y}$は$\theta_{X}$と同程度の性能を有するであろう。
その仮定のもとで、重力参照計の精度を偏光角較正の精度($\delta\theta_{\mathrm{grabity}}$)に焼き直すと、
望遠鏡が較正を行う基本姿勢$\mathrm{elevation}=50\tcdegree,\,\mathrm{boresight}=0\tcdegree$では
$\delta\theta_{\mathrm{grabity}}<0.06\tcdegree$と評価された。
これは先行研究の$\delta\theta_{\mathrm{grabity}}<0.3\tcdegree$よりも大幅に改善された値である。

先行研究では、ワイヤーのたわみがもう1つの大きな系統誤差の要因であった。
なぜなら、先行研究ではすべての評価を人力で行っており、多大な労力がかかっていたとともに
その評価結果に人依存のバイアスやふらつきを含む可能性があったからである。
この現状を受け、本研究ではワイヤーのたわみを自動で評価する装置を開発した。
たわみの長さが既知のワイヤーを測定することで評価精度として$<\SI{15}{\micro m}$を得た。
これは装置の光学系を考えた際に得る誤差とよく一致する。
さらに、この評価装置を用いて、実際に使用されるスパースワイヤーグリッドのたわみを評価した。
その評価結果はたわみによる偏光角への影響$\theta_{\mathrm{sag}}$にして、$\theta_{\mathrm{sag}}<0.02\tcdegree$であった。
この値は本装置を用いてたわみ角が大きいワイヤーを選別・張り直しを2回繰り返した結果であり、同じように張り直しの工程を繰り返すことで、
$\theta_{\mathrm{sag}}<0.02\tcdegree$のたわみ角を持つスパースワイヤーグリッドを作成することが可能であることを示した。

表\ref{tab:summary_systematic_error}に、本研究で評価した系統誤差の値を更新した、スパースワイヤーグリッドの装置由来の系統誤差をまとめる。
偏光角較正の系統誤差の合計$\delta\theta$は$\delta\theta<0.09\tcdegree$であった。
これはSimons Observatoryにおける要求精度$\delta\theta<0.1\tcdegree$を満たしている。

\begin{table}
    \centering
    \label{tab:summary_systematic_error}
    \caption{本論文で得た評価値を更新したスパースワイヤーグリッドの系統誤差のまとめ。}
    \begin{threeparttable}[H]
        \begin{tabular}{cllc}
            \hline\hline
            系統誤差要因 & 先行研究 & 本研究 & \quad \\
            \hline
            ワイヤー設置精度 & $<0.02\tcdegree$ & $<0.02\tcdegree$ & 先行研究を踏襲 \\
            エンコーダ精度 & $<0.03\tcdegree$ & $<0.03\tcdegree$ & \char"3003 \\
            エンコーダ零点 & $<0.04\tcdegree$ & $<0.04\tcdegree$ & \char"3003 \\
            重力参照計 & $<0.3\tcdegree$ & $<0.06\tcdegree$ & 本研究で改善\tnote{*} \\
            ワイヤーのたわみ & $<0.05\tcdegree$ & $<0.02\tcdegree$ & \char"3003 \\
            \hline
            合計 & $<0.3\tcdegree$ & $<0.09\tcdegree$ \\
            \hline\hline
        \end{tabular}
        \begin{tablenotes}[flushleft]\footnotesize
            \item[*] $X$軸の精度が$\delta\theta_{X}<0.04\tcdegree$と評価値であるのに対し、$Y$軸の精度は見込み精度として$\delta\theta_{Y}<0.04\tcdegree$と仮定。
        \end{tablenotes}
    \end{threeparttable}
\end{table}

\section{今後の展望}
今後の展望として、まず第一に初期不良で正常に動作していなかった重力参照計の$Y$軸の再評価が大切である。
この評価によりその誤差が$\delta\theta_{Y}<0.04\tcdegree$であることが実証されれば、偏光角較正の装置に起因する系統誤差は
疑いようもなく要求値である$\delta\theta<0.1\tcdegree$を満たすこととなる。

本論文で開発したワイヤーのたわみの自動評価装置の結果によれば、ワイヤーのたわみ量の大きさにはワイヤーの張る順番に起因する傾向が存在していた。
これは系統誤差として懸念するほど大きな量ではないものの、今後さらなる系統誤差の削減が必要になった場合にはワイヤーの張り方というところから改善する必要がある。
また、本装置は自動でたわみ量の評価が可能であるため、観測サイトにてスパースワイヤーグリッドのたわみを定期的に評価し、
その品質を維持するということも可能である。

Simons Observatoryでは現在稼働している3台のSATに加え、日本によるLF帯を観測するSAT(JSAT)の建設が予定されている。
現在稼働中の3台のSATには既にスパースワイヤーグリッドを用いた偏光角較正装置が搭載されているが、
そこで使用されている重力参照計を今回評価した新しいものと交換し、スパースワイヤーグリッドの部分を
本研究で開発したワイヤーのたわみの自動評価装置を用いて評価されたたわみの小さいものに交換すれば、高精度な偏光角較正を実現できる。
また、追加されるJSATにおいてもスパースワイヤーグリッドが開発され、導入される予定である。
本論文にて得られた系統誤差の知見を活かし、較正精度$\delta\theta<0.1\tcdegree$を満たす偏光角の較正が見込まれる。

以上のまとめをもって、Simons Observatoryは検出器の偏光角を$\delta\theta<0.1\tcdegree$で較正し、
前人未到の原始重力波由来のBモード偏光観測を目指す。
\end{document}