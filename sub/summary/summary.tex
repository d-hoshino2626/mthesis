\documentclass[../../main.tex]{subfiles}

\graphicspath{{../../fig/}}
\setcounter{section}{0}

\begin{document}
\chapter{まとめと今後の展望}
\section{まとめ}
宇宙マイクロ波背景放射(CMB)の温度揺らぎは、我々に$\Lambda \mathrm{CDM}$モデルという宇宙の進化を記述するモデルを与えた。
さらなる宇宙論の発展として、ビッグバンよりも以前に空間の指数関数的膨張が起こったとするインフレーションモデルが存在する。
CMBの偏光パターンを精査し、原始重力波由来のBモード偏光パターンを発見することは、インフレーションの直接的証拠となる。
地上CMB偏光観測実験であるSimons Observatory実験は、このBモードをテンソルスカラー比にして$\sigma(r)=0.003$の精度で観測することを目指している。
この精度を達成するためには、検出器の応答偏光方向(偏光角)を$\delta\theta<0.1\tcdegree$という精度で較正する必要がある。
そのために使用されるのが、スパースワイヤーグリッドを用いた偏光角較正装置である。



\section{今後の展望}
\end{document}