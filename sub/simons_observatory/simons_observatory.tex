\documentclass[../../main.tex]{subfiles}

\graphicspath{{../../fig/}}
\setcounter{section}{0}

\begin{document}
\chapter{Simons Observatory実験}
\section{Simons Observatory実験}
Simons Observatory実験 (以後、SOと呼ぶ) は、チリのアタカマ砂漠を拠点とする史上最大規模の地上CMB観測実験である。
現在、口径 $0.5\ \mathrm{m}$ の小口径望遠鏡 (Small Aperture Telescope, SAT) 3台と、
口径 $6\ \mathrm{m}$ の大口径望遠鏡 (Large Aperture Telescope, LAT) 1台を用いた観測が進められている。\cite{so:current_status}
検出器としてはTES (Transition Edge Sensor) 検出器を採用しており、SATにはそれぞれ約1万個ずつ、LATには約3万個の検出器が搭載されている。
合計約6万個もの検出器を通してCMBの変更を高精度で測定し、インフレーションに由来する原始重力波の検出や、
ニュートリノの有効世代数、ニュートリノ質量和の測定を目指す。\cite{so:science_forecast}

立体角 $\Omega$ 、開口面積 $A$、観測波長 $\lambda$ について、回折限界の関係式
\begin{equation}
    \Omega = \dfrac{\lambda^2}{A}
\end{equation}
を考えると、より大きな口径 $A$ を持つ望遠鏡ほどより高い角度分解能を有し、小角度の相関を観測するのに適していることがわかる。
その一方で、大口径の望遠鏡は一度に観測できる範囲も小さくなるため、大角度の相関を観測するのに時間を要し、大気揺らぎの影響を受けやすくなってしまう。
以上の理由から、小口径で大角度相関を調べるSATと、大口径で小角度相関を調べるLATを組み合わせることで、CMBのより精密な測定を実現する。

\section{Large Aperture Telescope (LAT)}
\section{Small Aperture Telescope (SAT)}
\subsection{TES検出器}

\subsection{極低温連続回転式半波長板 (HWP)}
大気による熱放射は常に揺らいでいる。
これは大気による $1/f$ ノイズとして知られ、CMB偏光観測実験においては、このノイズとCMB偏光信号を分離することが重要である。
Simons Observatoryでは、この大気による熱放射を取り除くために、極低温連続回転式半波長板(cryogenic continuously rotating Half-Wave Plate, 以後、単にCHWPと呼ぶ)を用いる。\cite{so:hwp_yamada}

一般に、HWPは複屈折の特性を持つ素材からなり、素子中のある決まった軸に対して電場成分を反転させる。
すなわち、HWPに入射する光の電場 $\bm{E}$ はHWPを通過することで
\begin{align}
    E_{1} &= E_{1} \\
    E_{2} &= -E_{2}
\end{align}
となる。ここで、$1, 2$ はそれぞれHWPの光学軸を表し、1軸に対して電場成分が反転している。
入射光として偏光角がHWPの1軸から測って $\chi$ であるような直線偏光した光を考える。
HWPを通過した後の偏光角は $-\chi$ となり、偏光が1軸対称に反転、つまり $-2\chi$ だけ変化する。(図\ref{fig:so-hwp_satoru})
この性質により、入力信号のストークスパラメータがそれぞれ $I_{\mathrm{in}}(t), Q_{\mathrm{in}}(t), U_{\mathrm{in}}(t)$ であるとき、出力信号 $d_m(t)$ は
\begin{equation}
    d_{\mathrm{m}}(t) = I_{\mathrm{in}}(t) + \varepsilon\Re\qty[\qty(Q_{\mathrm{in}}(t)+iU_{\mathrm{in}}(t))\exp(-i 4\chi)]
\end{equation}
となる。ここで、$\varepsilon$ は変調効率である。SOでは、HWPを $2\ \mathrm{Hz}$ で回転させることで、連続的に入射する直線偏光による信号を $8\ \mathrm{Hz}$ に変調して出力する。
HWPの角振動数を $\omega_{\mathrm{HWP}}$ とすると、$\chi = \omega_{\mathrm{HWP}}t$ と表され、出力信号は
\begin{equation}
    d_{\mathrm{m}}(t) = I_{\mathrm{in}}(t) + \varepsilon\Re\qty[\qty(Q_{\mathrm{in}}(t)+iU_{\mathrm{in}}(t))\exp(-i 4\omega_{\mathrm{HWP}}t)]
\end{equation}
となる。検出器はある偏光角方向 $\theta_{\mathrm{det}}$ にのみ感度を持つため、最終的に検出器が読み出す信号 $d_{\mathrm{m}, \mathrm{det}}$ は
\begin{equation}
    \label{eq:so-hwp_modulation}
    d_{\mathrm{m}, \mathrm{det}}(t) = I_{\mathrm{in}}(t) + \varepsilon\Re\qty[\qty(Q_{\mathrm{in}}(t)+iU_{\mathrm{in}}(t))\exp\qty{-i \qty(4\omega_{\mathrm{HWP}}t - 2\theta_{\mathrm{det}})}]
\end{equation}
となる。この信号のフーリエ変換は
\begin{equation}
    \begin{split}
        \tilde{d}_{\mathrm{m}, \mathrm{det}}(\Omega) = &\tilde{I}_{\mathrm{in}}(\Omega) \\
            &+ \dfrac{\varepsilon}{2}\qty[\qty{\tilde{Q}_{\mathrm{in}}(\Omega+4\omega_{\mathrm{HWP}})+i\tilde{U}_{\mathrm{in}}(\Omega+4\omega_{\mathrm{HWP}})}\exp\qty(i2\theta_{\mathrm{det}})] \\
            &+ \dfrac{\varepsilon}{2}\qty[\qty{\tilde{Q}_{\mathrm{in}}(\Omega-4\omega_{\mathrm{HWP}})-i\tilde{U}_{\mathrm{in}}(\Omega-4\omega_{\mathrm{HWP}})}\exp\qty(i2\theta_{\mathrm{det}})]
    \end{split}
\end{equation}
である。この式はほとんど時間変化しない信号$(\Omega\sim0)$がHWPを通過することで、周波数 $\pm 4\omega_{\mathrm{HWP}}$ のところに移ることを示している。
このようにして、元々 $1/f$ ノイズが大きかった低周波帯の信号を、ノイズの少ない高周波帯に変換できる。
$Q_{\mathrm{in}}+iU_{\mathrm{in}}$ を得るためには、$+4\omega_{\mathrm{HWP}}$ のまわりのみを通すバンドパスフィルタ $\mathcal{F}^{\mathrm{BPF}}$ を通した後、2倍して位相を元に戻せば良い。
つまり、復調後に得られる信号 $d_{\mathrm{d, det}}$ は
\begin{align}
    d_{\mathrm{d, det}}(t) &= \mathcal{F}^{\mathrm{BPF}}\qty[d_{\mathrm{m,det}}(t)]\times 2\exp\qty(i4\omega_{\mathrm{HWP}}t) \\
    &= \varepsilon\qty[Q_{\mathrm{in}}(t) + iU_{\mathrm{in}}(t)]\exp\qty[i2\theta_{\mathrm{det}}]
\end{align}
となっている。

\begin{figure}[H]
    \centering
    \includegraphics[width=0.8\textwidth]{simons_observatory/hwp_satoru.pdf}
    \caption{HWPを通過することで、偏光角が変化することを示した概念図。青い軸が1軸、緑の軸が2軸に対応する。
    入射した直線偏光の偏光角が1軸に対して $\chi$ であり、複屈折によって $-2\chi$ だけ変化する。}
    \label{fig:so-hwp_satoru}
\end{figure}

\section{偏光角較正の重要性とその手法}
CMB偏光観測実験でのB-mode観測において、検出器の偏光角を精度よく知ることは極めて重要である。
本節では、偏光角の誤較正が観測されたB-mode偏光にどのような影響を及ぼすかを説明し、
CMB偏光観測実験においてどのような偏光角較正手法が歴史的に取られてきたかについて述べる。
\subsection{偏光角の誤較正に伴う偏光の漏れ込み}
今、すべての検出器における偏光角を $\delta \theta$ だけ誤って較正してしまったとすると、
観測されるストークスパラメータ $Q_{\mathrm{obs}},\ U_{\mathrm{obs}}$ は真のストークスパラメータ $Q,\ U$ に対して
\begin{align}
    Q_{\mathrm{obs}} \pm iU_{\mathrm{obs}} &= e^{\pm i2\delta\theta}\qty(Q \pm iU) \\
    &= e^{\pm i2\delta\theta}\sum_{l=2}\sum_{m=-l}^{l} {}_{\pm 2}a_{lm}\ {}_{\pm 2}Y_{lm}(\theta, \phi)
\end{align}
と表される。\cite{so:Keeting_2023}\cite{so:Kaufman_2014}
Eモード、Bモード偏光を表現する係数 $a_{lm}^{E}, a_{lm}^{B}$ を用いると、
スピン2の球面調和関数で展開する際の係数 ${}_{\pm 2}a_{lm}$ は
\begin{equation}
    {}_{\pm 2}a_{lm} = -\qty(a_{lm}^{E} \pm i a_{lm}^{B})
\end{equation}
であるから
\begin{align}
    Q_{\mathrm{obs}} \pm iU_{\mathrm{obs}} &= \sum_{l=2}\sum_{m=-l}^{l} \qty[-\qty(a_{lm}^{E} \pm i a_{lm}^{B})]e^{\pm i2\delta\theta}\ {}_{\pm 2}Y_{lm}(\theta, \phi) \\
    &= \sum_{l=2}\sum_{m=-l}^{l} \qty[-\qty(a_{lm,\,\mathrm{obs}}^{E} \pm i a_{lm,\,\mathrm{obs}}^{B})] {}_{\pm2}Y_{lm}(\theta, \phi)
\end{align}
が観測されることとなる。したがって、パワースペクトル
\begin{equation}
    C_{l}^{XX'} = \dfrac{1}{2l+1}\sum_{m=-l}^{l}\left\langle \qty(a_{lm}^{X})^{*} a_{lm}^{X'} \right\rangle
\end{equation}
\begin{equation}
    C_{l}^{XX'} = \dfrac{1}{2l+1}\sum_{m=-l}^{l}\left\langle a_{lm}^{*\,X} a_{lm}^{X'} \right\rangle
\end{equation}
は、偏光角の誤較正によって
\begin{equation}
    \mqty( C_{\mathrm{obs},\,l}^{TT} \\[1ex]
           C_{\mathrm{obs},\,l}^{TE} \\[1ex]
           C_{\mathrm{obs},\,l}^{TB} \\[1ex]
           C_{\mathrm{obs},\,l}^{EE} \\[1ex]
           C_{\mathrm{obs},\,l}^{EB} \\[1ex]
           C_{\mathrm{obs},\,l}^{BB} )
    = \mqty( 1 & 0 & 0 & 0 & 0 & 0 \\[1ex]
             0 & \cos\qty(2\delta\theta) & 0 & -\sin\qty(2\delta\theta) & 0 & 0 \\[1ex]
             0 & 0 & \cos^2\qty(2\delta\theta) & 0 & -\sin\qty(4\delta\theta) & \sin^2\qty(2\delta\theta) \\[1ex]
             0 & \sin\qty(2\delta\theta) & 0 & \cos\qty(2\delta\theta) & 0 & 0 \\[1ex]
             0 & 0 & \sin\qty(4\delta\theta)/2 & 0 & \cos\qty(4\delta\theta)/2 & -\sin\qty(4\delta\theta)/2 \\[1ex]
             0 & 0 & 0 & \sin^2\qty(2\delta\theta) & \sin(4\delta\theta) & \cos^2\qty(2\delta\theta) \\[1ex] )
    \mqty( C_{l}^{TT} \\[1ex]
              C_{l}^{TE} \\[1ex]
              C_{l}^{TB} \\[1ex]
              C_{l}^{EE} \\[1ex]
              C_{l}^{EB} \\[1ex]
              C_{l}^{BB} ) 
\end{equation}

\subsection{偏光角較正手法}
\subsubsection{Tauras A (Tau A)}
\subsubsection{月}
\subsubsection{誘電体シート}
\subsubsection{デンスワイヤーグリッド}
\subsubsection{スパースワイヤーグリッド}
\end{document}