\documentclass[../../main.tex]{subfiles}

\graphicspath{{../../fig/}}
\setcounter{section}{0}

\begin{document}
\chapter{スパースワイヤーグリッドを用いた偏光角較正装置}
\section{概要}
スパースワイヤーグリッドの外観を図\ref{hogehoge}に示す。
これは金属製のワイヤーを、入射光よりも十分長い間隔で平行に張ったものであり、ワイヤー軸に沿った偏光を生成する。
Simons Observatory実験では、検出器の偏光角較正のために、人工偏光光源としてスパースワイヤーグリッドを用いた偏光角較正装置を使用する。

\section{スパースワイヤーグリッド}
これはアルミニウム製の直径830mmの円環に、直径0.1mmのタングステンワイヤーを20mm間隔で張り巡らせたものである。

\section{偏光信号の生成原理}
金属製のワイヤーが、周囲から来た入射光を反射することを考える。
入射光の波長がワイヤーの直径よりも十分に長い場合、ワイヤー中の自由電子はワイヤーに沿う方向のみに動くと見なすことができ、ワイヤーは自身の軸に沿った偏光状態を持つ光のみを反射する。
このようなワイヤーを望遠鏡の視野に置くと、ワイヤーは周囲の環境から来る熱放射を反射し、ワイヤー軸と同じ方向に偏光した光を望遠鏡に送り込む。
実際には望遠鏡は空も視野に含み、無偏光な大気放射、ワイヤーからの偏光信号の重ね合わせが見える。
後述する回転半波長板という光学素子を用いることで、無偏光な大気放射を取り除き、ワイヤーからの偏光信号のみを抽出して偏光角較正に用いる。
また、ワイヤー間隔を調整することで実行的な放射温度を調整し、CMB望遠鏡の検出器に入射する偏光光の強度を調整することができる。

\section{偏光角較正の原理}

\section{設計}
\section{}

\end{document}