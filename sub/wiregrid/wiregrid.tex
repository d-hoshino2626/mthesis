\documentclass[../../main.tex]{subfiles}

\graphicspath{{../../fig/}}
\setcounter{section}{0}

\begin{document}
\chapter{スパースワイヤーグリッドを用いた偏光角較正装置}
SO
\section{概要}
スパースワイヤーグリッドの外観を図\ref{hogehoge}に示す。
これは金属製のワイヤーを、入射光よりも十分長い間隔で平行に張ったものであり、ワイヤー軸に沿った偏光を生成する。
Simons Observatory実験では、検出器の偏光角較正のために、人工偏光光源としてスパースワイヤーグリッドを用いた偏光角較正装置を使用する。

\section{スパースワイヤーグリッド}
これはアルミニウム製の直径830mmの円環に、直径0.1mmのタングステンワイヤーを20mm間隔で張り巡らせたものである。

\section{偏光信号の生成原理}
金属製のワイヤーが、周囲から来た入射光を反射することを考える。
入射光の波長がワイヤーの直径よりも十分に長い場合、ワイヤー中の自由電子はワイヤーに沿う方向のみに動くと見なすことができ、ワイヤーは自身の軸に沿った偏光状態を持つ光のみを反射する。
このようなワイヤーを望遠鏡の視野に置くと、ワイヤーは周囲の環境から来る熱放射を反射し、ワイヤー軸と同じ方向に偏光した光を望遠鏡に送り込む。
実際には望遠鏡は空も視野に含み、無偏光な大気放射、ワイヤーからの偏光信号の重ね合わせが見える。
\ref{}にて述べた、回転半波長板という光学素子を用いることで無偏光な大気放射を取り除き、ワイヤーからの偏光信号のみを抽出して偏光角較正に用いる。
また、ワイヤー間隔を調整することで実行的な放射温度を調整し、CMB望遠鏡の検出器に入射する光の強度を調整することができる。

\section{偏光角較正の原理}
式\eqref{eq:so-hwp_modulation}において、入射光としてワイヤー由来の偏光角 $\theta_{\mathrm{WG}}$ の直線偏光した光を考える。
$Q_{\mathrm{in}}(t) + iU_{\mathrm{in}}(t) = \exp\qty[2i\theta_{\mathrm{WG}}]$ となるため、
\begin{equation}
    d_{\mathrm{m}, \mathrm{det}}(t) = I_{\mathrm{in}}(t) + \varepsilon\Re\qty[\exp\qty{-i \qty(4\omega_{\mathrm{HWP}}t - 2\theta_{\mathrm{det}} - 2\theta_{\mathrm{WG}})}]
\end{equation}
となる。
ワイヤー由来の光の強度はほとんど時間変化しないため、$I_{\mathrm{in}}(t) \simeq \mathrm{const.}$ とみなせる。
したがって、この変調信号は時系列データとして位相オフセット $2(\theta_{\mathrm{det}} + \theta_{\mathrm{WG}})$ を持った
角振動数 $4\omega_{\mathrm{HWP}}$ の正弦波としてみえる。
理想的な時系列データのイメージを図\ref{}に示す。
これを復調することで、式\eqref{eq:so-hwp_demod}より
\begin{equation}
    d_{\mathrm{d, det}} = \varepsilon\exp\qty[i2\qty(\theta_{\mathrm{WG}} + \theta_{\mathrm{det}})]
\end{equation}
という偏光情報のみを得る。
ワイヤーの角度 $\theta_{\mathrm{WG}}$ を変化させると、この $d_{\mathrm{d, det}}$ は複素平面上で円(較正円と呼ぶ)を描く。
実際には、$\theta_{\mathrm{WG}}$ を $22.5^{\circ}$ ごとに変化させ、較正円を描く。
較正円上では、


\section{設計}

\section{}

\end{document}